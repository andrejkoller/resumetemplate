\documentclass[letterpaper,11pt]{article}

\usepackage{latexsym}
\usepackage[empty]{fullpage}
\usepackage{titlesec}
\usepackage{marvosym}
\usepackage[usenames,dvipsnames]{color}
\usepackage{verbatim}
\usepackage{enumitem}
\usepackage[hidelinks]{hyperref}
\usepackage{fancyhdr}
\usepackage[english]{babel}
\usepackage{tabularx}
\usepackage{fontawesome5}
\usepackage{multicol}
\setlength{\multicolsep}{-3.0pt}
\setlength{\columnsep}{-1pt}
\input{glyphtounicode}
\usepackage[margin=1.4cm]{geometry}

\pagestyle{fancy}
\fancyhf{}
\fancyfoot{}
\renewcommand{\headrulewidth}{0pt}
\renewcommand{\footrulewidth}{0pt}

\urlstyle{same}

\raggedbottom
\raggedright
\setlength{\tabcolsep}{0in}

\titleformat{\section}{
  \vspace{-4pt}\scshape\raggedright\large\bfseries
}{}{0em}{}[\color{black}\titlerule \vspace{-5pt}]

\pdfgentounicode=1

% Custom commands
\newcommand{\resumeItem}[1]{
  \item\small{
    {#1 \vspace{0pt}}
  }
}

\newcommand{\classesList}[4]{
  \item\small{
    {#1 #2 #3 #4 \vspace{-2pt}}
  }
}

\newcommand{\resumeSubheading}[4]{
  \vspace{-2pt}\item
  \begin{tabular*}{1.0\textwidth}[t]{l@{\extracolsep{\fill}}r}
    \textbf{#1} & \textbf{\small #2} \\
    \textit{\small#3} & \textit{\small #4} \\
  \end{tabular*}\vspace{-7pt}
}

\newcommand{\resumeSubSubheading}[2]{
  \item
  \begin{tabular*}{0.97\textwidth}{l@{\extracolsep{\fill}}r}
    \textit{\small#1} & \textit{\small #2} \\
  \end{tabular*}\vspace{-7pt}
}

\newcommand{\resumeProjectHeading}[2]{
  \item
  \begin{tabular*}{1.001\textwidth}{l@{\extracolsep{\fill}}r}
    \small#1 & \textbf{\small #2} \\
  \end{tabular*}\vspace{-7pt}
}

\newcommand{\resumeSubItem}[1]{\resumeItem{#1}\vspace{-4pt}}

\renewcommand\labelitemi{$\vcenter{\hbox{\tiny$\bullet$}}$}
\renewcommand\labelitemii{$\vcenter{\hbox{\tiny$\bullet$}}$}

\newcommand{\resumeSubHeadingListStart}{\begin{itemize}[leftmargin=0.0in, label={}]}
\newcommand{\resumeSubHeadingListEnd}{\end{itemize}\vspace{0pt}}
\newcommand{\resumeItemListStart}{\begin{itemize}}
\newcommand{\resumeItemListEnd}{\end{itemize}\vspace{-5pt}}

\begin{document}

%----------HEADING----------
\begin{center}
  {\Large \scshape Andrej Koller} \\[2mm]
  \footnotesize
  \faMapPin \ Johann-Riederer-Straße 44, 94036 Passau ~
  \faPhone\ (+49) 171 5135649 ~ 
  \faEnvelope\ \href{mailto:andrejkoller@outlook.com}{andrejkoller@outlook.com}
\end{center}

\begin{center}
  \footnotesize
  \faLinkedin\ \underline{\href{https://www.linkedin.com/in/andrejkoller}{linkedin.com/in/andrejkoller}} ~
  \faGithub\ \underline{\href{https://github.com/andrejkoller}{github.com/andrejkoller}} ~
  \faGlobe\ \underline{\href{https://www.andrejkoller.com}{www.andrejkoller.com}}
\end{center}

%-----------SUMMARY-----------
\section{Professionelle Zusammenfassung}
Engagierter Anwendungsentwickler mit Fokus auf moderne Webtechnologien, Benutzeroberflächengestaltung und hervorragende Nutzererfahrung. Mein Ziel ist es, intuitive und performante Anwendungen zu entwickeln, die sowohl funktional als auch visuell ansprechend überzeugen.

%-----------EDUCATION-----------
\section{Bildungsweg}
\resumeSubHeadingListStart
  \resumeSubheading
    {Berufsschule Passau}{Sep. 2022 – Jan. 2025}
    {Ausbildung zum Fachinformatiker für Anwendungsentwicklung}{Abschluss: Jan. 2025}
\resumeSubHeadingListEnd
\resumeItemListStart
  \resumeItem{Praktische Erfahrungen in der Entwicklung mit HTML5, CSS3, SCSS, JavaScript, TypeScript, Angular, .NET, C, C++, Java, MSSQL, TailwindCSS, Bootstrap, WordPress}
\resumeItemListEnd

\resumeSubHeadingListStart
  \resumeSubheading
    {Technische Hochschule Deggendorf}{Okt. 2019 – März 2022}
    {Betriebswirtschaft und Angewandte Informatik (ohne Abschluss)}{}
\resumeSubHeadingListEnd

\resumeSubHeadingListStart
  \resumeSubheading
    {Fachoberschule Passau}{Sep. 2016 – Juli 2019}
    {Fachabitur, Fachrichtung Wirtschaft und Verwaltung}{Fachhochschulreife}
\resumeSubHeadingListEnd

%-----------PROJECTS-----------
\section{Projekte}
\resumeSubHeadingListStart

  \resumeProjectHeading{\textbf{Sinai Workplace App} $|$ \emph{\href{https://github.com/andrejkoller}{GitHub}}}{Angular $|$ .NET $|$ MSSQL}
  \resumeItemListStart
    \resumeItem{Entwicklung einer Webanwendung zur Buchung und Verwaltung von Arbeitsplätzen in Großraumbüros.}
    \resumeItem{Verwendet moderne Angular-Komponenten auf Basis der Angular-Standalone-Komponentenstruktur, verbunden mit einer .NET-API und MSSQL-Datenbank.}
    \resumeItem{Echtzeitstatus über SignalR zur Anzeige von Belegungen und freien Plätzen.}
  \resumeItemListEnd

  \resumeProjectHeading{\textbf{Linksheet – Custom Link App} $|$ \emph{\href{https://github.com/andrejkoller}{GitHub}}}{React $|$ .NET $|$ MSSQL}
  \resumeItemListStart
    \resumeItem{Entwicklung eines Tools zur Organisation und Anpassung eigener Links und Ressourcen.}
    \resumeItem{Nutzer können eigene Links erstellen, bearbeiten und je nach Bedarf aktivieren oder deaktivieren.}
  \resumeItemListEnd

  \resumeProjectHeading{\textbf{Persönliche Webseiten – Portfolio und Musikseite}  $|$ \emph{\href{https://github.com/andrejkoller}{GitHub}}}{React $|$ Next.js $|$ GSAP}
  \resumeItemListStart
    \resumeItem{\textbf{andrejkoller.com}: Private Website zur Präsentation klassischer Musik (Piano und Orgel), realisiert mit Next.js, responsive und für verschiedene Endgeräte optimiert.}
    \resumeItem{\textbf{work.andrejkoller.com}: Portfolioseite mit Überblick über abgeschlossene und laufende Coding-Projekte. GSAP sorgt für weiche Animationen und interaktive Übergänge.}
  \resumeItemListEnd

\resumeSubHeadingListEnd

%-----------SKILLS-----------
\section{IT-Kenntnisse}
\begin{itemize}[leftmargin=0.15in, label={}]\small{\item{
  \textbf{Programmiersprachen}{: HTML5, CSS3, SCSS, JavaScript, TypeScript, C\#} \\[1mm]
  \textbf{Frameworks \& Tools}{: Angular, React, Vue, .NET, MSSQL, Node.js, TailwindCSS, Bootstrap} \\[1mm]
  \textbf{Entwicklungsumgebungen}{: Visual Studio Code, Visual Studio, Postman, Git} \\[1mm]
  \textbf{CMS / UI-Design}{: WordPress, Webflow, Figma, Canva}
}}\end{itemize}

%-----------LANGUAGES-----------
\section{Sprachkenntnisse}
\begin{itemize}[leftmargin=0.15in, label={}]\small{\item{
  \textbf{Deutsch}{: Muttersprache} \\[1mm]
  \textbf{Russisch}{: Muttersprache} \\[1mm]
  \textbf{Englisch}{: Sehr gute Kenntnisse in Wort und Schrift (C1)} \\[1mm]
}}\end{itemize}
\end{document}
